
OBJECTIVE: The objective is to produce a blockchain framework that has the capability to preserve mission-critical enterprise data during a catastrophic physical attack on the United States.

DESCRIPTION: This project seeks to support the Department of Defense (DoD) Continuity of Operations (COOP) by developing information technology frameworks capable of preserving mission critical data despite catastrophic nuclear, electromagnetic pulse (EMP), and/or cyberwarfare attacks. [1] [2] Specifically, we plan to apply quantum-resistant blockchain technologies to protect and preserve essential mission-critical enterprise data in scenarios where a large-scale weapons of mass destruction (WMD) attack imperils the physical survival of most DoD information systems. 
Blockchain is a set of technologies for creating a distributed ledger of validated data blocks chained together. The creation of a block in a blockchain is a secure process, and once the addition of the data block to the blockchain is complete, the data block is both immutable and auditable. [3] There is emerging work that applies the original design principles of the Internet to producing inter-operable, robust blockchain systems that are resilient in the face of some cyber-attacks (e.g., distributed denial-of-service). [4] 
It is reasonable to assume that possible large nuclear and/or EMP attacks will be coupled with a cyberwarfare offensive aimed at disrupting potential defenses and neutralizing countermeasures. Quantum computing is an emerging technology that will revolutionize cyberwarfare both offensively and defensively when it appears at scale. [5] Quantum computing utilizes quantum phenomena using qubits and qubit gates. [6] By applying algorithms that exploit the special properties of qubit gates, quantum computing systems are expected to compromise many existing encryption schemes including those used in Bitcoin, a well-known blockchain system. [7] [8] Researchers have developed encryption schemes that are expected to be more resistant to quantum attacks and have begun incorporating these quantum-resistant encryption schemes into new blockchain frameworks. [9] [10] Use of quantum-resistant encryption is therefore essential in new systems that support COOP. 
Resilient blockchain frameworks that support COOP will need to operate in primarily two different network environments: 1) networks with medium-latency and medium-bandwidth exemplified by global Wide Area Networks (WANs), 2) networks with high-latency and low-bandwidth exemplified by WANs that include satellite links. [11]

PHASE I: An engineering prototype will be constructed using a quantum-resistant blockchain framework designed to support COOP. The engineering prototype will be comprised from a set of twenty independent systems connected logically through the blockchain framework. The performer will conduct a series of experiments on the engineering prototype while simulating the two different network environments outlined above. In the experiments, simulated users will stochastically generate a large number of documents that are passed to one of the blockchain systems for replication to the other blockchain systems. For the simulated global WAN, these documents should vary in size from 20 kB to 10 MB. For the WAN that includes a simulated satellite link, these documents should vary in size from 20 kB to 500 kB. In addition to the network factor, the performer’s experimental design should include factors for: • degree of sudden physical compromise of the system (i.e., a percentage of systems whose total sudden physical loss is simulated), • degree of cyber-compromise (i.e., a percentage of systems whose compromise is simulated), and • degree of denial-of-service (i.e., levels of a cyber adversary’s denial-of-service efforts). The performer will observe the accuracy of the preserved documents as well as the overall recall of the generated documents at the end of each experiment. The performer will also collect the total quantity and overall characteristics of the network traffic generated by the system. The performer will verify and validate that data block generation is secure and that the generated data blocks are immutable and auditable. The phase I deliverable is a report, delivery of collected data, and a demonstration of the engineering prototype system.

PHASE II: The performer will investigate how to integrate the resilient blockchain framework with existing DoD COOP infrastructure. An advanced prototype of the resilient blockchain framework designed for integration with existing DoD COOP infrastructure will be developed. The advanced prototype will be comprised from a set of two hundred independent systems connected logically through the blockchain framework. The advanced prototype will address any shortcomings discovered in the engineering prototype. The performer will conduct a similar set of experiments with document generation occurring at a scale ten times greater than Phase I experiments. The Phase II experiments should also include experiments where simulation of cyber-attacks and the sudden physical compromise of the system are staggered on a scale of minutes to hours. The performer will again observe the accuracy and recall of the preserved documents as well as the network traffic characteristics. The performer will verify and validate the security, immutability, and auditability of the total advanced prototype.

PHASE III DUAL-USE APPLICATIONS: Finalize and commercialize developed quantum-resistant blockchain frameworks for use by DoD (specifically Office of Secretary of Defense CIO and Defense Information Systems Agency) and potentially other government customers. Although additional funding may be provided through DoD sources, the awardee should look to other public or private sector funding sources for assistance with transition and commercialization.
