\label{sec:related}
In this section, we present a brief summary of related work and how they contribute towards
our design and implementation.


\subsection{Existing work on distributed systems}

The ability (and often requirement) to communicate over a shared channel is a defining characteristic of distributed
programs, and many of the key results in the field pertain to the possibility and impossibility of performing
distributed computations under particular sets of network conditions.

For example, the celebrated FLP impossibility result9 demonstrates the inability to guarantee consensus in an
asynchronous network (i.e., one facing indefinite communication partitions between processes) with one faulty process.
This means that, in the presence of unreliable (untimely) message delivery, basic operations such as modifying the set
of machines in a cluster (i.e., maintaining group membership, as systems such as Zookeeper are tasked with today) are
not guaranteed to complete in the event of both network asynchrony and individual server failures. Related results
describe the inability to guarantee the progress of serializable transactions\cite{Davidson1985}, linearizable
reads/writes \cite{Gilbert2002}, and a variety of useful, programmer-friendly guarantees under adverse conditions
\cite{Bailis2013}. The implications of these results are not simply academic: these impossibility results have motivated
a proliferation of systems and designs offering a range of alternative guarantees in the event of network failures.5
Under a friendlier, more reliable network that guarantees timely message delivery, however, FLP and many of these
related results no longer hold:8 by making stronger guarantees about network behavior, we can circumvent the
programmability implications of these impossibility proofs.

The Ubiq \cite{ubiq}framework fully tolerates infrastructure degradation and datacenter-level outages without any manual
intervention. It also guarantees exactly-once semantics for application pipelines to process logs in the form of event
bundles. Ubiq has been in production for Google’s advertising system for many years and has served as a critical log
processing framework for hundreds of pipelines. Our production deployment demonstrates linear scalability with machine
resources, extremely high availability even with underlying infrastructure failures, and an end-to-end latency of under
a minute.

\subsection{Blockchain systems}
Over the last few years, there have been a large number of projects who have secured funding
to work on the performance aspect of blockchain while focussed on a single class of attack
called the Sybil attack \cite{john2002}. The notable designs include \cite{rapidchain, chainspace, ava2018, thundertoken} and Facebook's own Libra system (libra.org).

For permissioned blockchains, notable designs inlude the Hyperledger project and the Ethereum Enterprise project. We are
able to leverage learnings from both clases of projects to create a design that works best for the failure and
attack scenarios in order to preserve COOP for mission critical systems.

\subsection{Quantum resistant ledgers}

Quantum computers pose a security risk to current crytographic systems \cite{dasgupta2019}. As a result, many
encryption, authentication and key distribution schemes are under risk. There is an active body of research with solid
initial results that is focussed on building quantum-resistant cyptographic techniques \cite{qrl2016}. Our blockchain design
uses such primitives from the grounds-up to design quantum-resistant distributed ledgers as the core building block.

\subsection{Use of Zero Knowledge proofs}

Zero-knowledge proofs (zk-proofs \cite{bunz2018}) are an exciting new technoloy that enables proof of a statement without revealing any additional
knowledge to the verifier beyond the statement being proved. There are a wide range of applications possible for such
a technology in the commercial space. We envision using zk-proofs for secure commitments, storage proofs and validity of
commands in our system designed for COOP for mission critical operations. This will result in our system being strong
against quantum computer based attacks for the forseeable future.

\subsection{Blockchain as a Resilient Computer}

While not an explicit goal for Phase I or Phase II, the research and engineering work into using a blockchain-based
framework for COOP brings about the possibility of building a smart-contract based robust and attack-resistant computing
platform. Similar projects have been proposed in the literature, as a blockchain based computer for the world
\cite{reyes2018, yahya2019, hanke2018}. This opens up amazing new possibilities for a attack-resistant, failure
resistant blockchain computer that has the ability to protect missio critical data including taking actions as directed
by people in charge.

In my view, this would be an exciting area of research and developement for the Department of Defense to pursue after the completion
of Phase I and II.

