In this section, we present a brief summary of related work and how they contribute towards
our design and implementation.


\subsection{Existing work on distributed systems}

The ability (and often requirement) to communicate over a shared channel is a defining characteristic of distributed
programs, and many of the key results in the field pertain to the possibility and impossibility of performing
distributed computations under particular sets of network conditions.

For example, the celebrated FLP impossibility result9 demonstrates the inability to guarantee consensus in an
asynchronous network (i.e., one facing indefinite communication partitions between processes) with one faulty process.
This means that, in the presence of unreliable (untimely) message delivery, basic operations such as modifying the set
of machines in a cluster (i.e., maintaining group membership, as systems such as Zookeeper are tasked with today) are
not guaranteed to complete in the event of both network asynchrony and individual server failures. Related results
describe the inability to guarantee the progress of serializable transactions,7 linearizable reads/writes,11 and a
variety of useful, programmer-friendly guarantees under adverse conditions.3 The implications of these results are not
simply academic: these impossibility results have motivated a proliferation of systems and designs offering a range of
alternative guarantees in the event of network failures.5 Under a friendlier, more reliable network that guarantees
timely message delivery, however, FLP and many of these related results no longer hold:8 by making stronger guarantees
about network behavior, we can circumvent the programmability implications of these impossibility proofs.

Follow up with references from here: https://queue.acm.org/detail.cfm?id=2655736
