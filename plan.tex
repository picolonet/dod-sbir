\label{sec:plan}
This section presents a breakdown of the work planned for Phase I of this SBIR proposal. The work plan expands upon the
goals given in the original proposal with details on execution. 

\subsection{Simulation environment:}

As given in the call for proposals, we shall use two different environments for our simulations:

\begin{enumerate}
    \item {\bf Medium latency Wide-area networks :} We shall utilize existing research \cite{gustaf2013, haq2017}
to emulate networks with medium-latency and medium-bandwidth. Our testbed shall consist of 
containers that use virtualized network stacks on the Linux platform. We shall programmatically
emulate different network conditions to study the resilience of the blockchain system.

\item {\bf High latency WANs:}  We shall programmatically emulate networks with high-latency and low-bandwidth
exemplified by WANs that include satellite links. Our emulation testbed gives us the ability to also test for
coordinated link and node failures attacks such as that can happen in a disaster scenario as discussed next.
 
\item {\bf Other failure modes:} Apart from network failures, we shall also study the performance of the blockchain
system under coordinated models of attack by building upon the past research in this area \cite{xiao2006, niu2017,
kurar2014, zhou2010}. The specific failure/attack models include sudden coodinated failures of network links and nodes,
compromise of systems including security credentials and denial-of-service.

\end{enumerate}

Next, we discuss the specific milestones for Phase I for our work. 

\subsection{Phase I milestones:}

We present a series of milestones for our work in Phase I. 

\begin{enumerate}
\item {\bf Prototype Design:} Complete a design of a blockchain-based quantum resistant system for
mission critical operation and COOP goals. The milestone consists of a design document with high-level specification of
each component of the blockchain system, including design of the ledger, the consensus mechanism, database replication,
data query and other supported operations.


\item {\bf Internal Testbed:} Build a testbed using hosted containers and simulated networks for iterated model of code
development and testing before moving to the real test. This capability will be important for rapid prototyping, testing
and development in-house. This will also allow the team to test the blockchain with various synthetic failure models
before proceeding forward with the Phase I simulations.

\item {\bf Engineering prototype:} The prototype blockchain would have undergone enough internal testing to be ready for
the simulation run. This milestone also includes implemenation of the simulation environment, including creation of
synthetic users, test cases, instrumentation logic measuring real-time and statistical performance properties and
related software components. As given in the topic announcement, the simulation consists of conducting a set of
experiments to measure variuos aspects of the blockchain system. 

The engineering prototype will be constructed using a quantum-resistant blockchain framework designed to
support COOP. The engineering prototype will be comprised from a set of twenty independent systems connected logically
through the blockchain framework. 

\item {\bf Phase I Simulation:}  Our simulation will generate a stochastically significant number of documents mapped to
a statistical distribution of users. We shall draw from user distribution studies in distributed file systems research
to program specific access patterns. During the simulation, we shall induce failures as discussed above. The metrics
collected will include data on distribution of the file requests, details on failures and their impact on the system. We
shall study the effect on the blockchain system to ensure performance degrades gracefully with increased failure.

\end{enumerate}

All of the above work shall be documented in a report which will be made available upon completion of Phase I.
