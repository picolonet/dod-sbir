
\subsection{Smart contracts}

How smart contracts can be used to:

 - Trigger automated backups.
 - Activate recovery or "disaster" mode.
 - Allow for smart access control.
 - Other disaster recovery applications can be 
 built using smart contracts. Such as automated download
 to a secure location, cutting off access to certain areas
 as given by policy, etc.

 blurb on smart contracts: 

 The term smart contract dates to 1994, defined by Nick Szabo as “a computerized transaction
protocol that executes the terms of a contract. The general objectives of smart contract design are
to satisfy common contractual conditions (such as payment terms, liens, confidentiality, and even
enforcement), minimize exceptions both malicious and accidental, and minimize the need for
trusted intermediaries.” [17].
Smart contracts extend and leverage blockchain technology. A smart contract is a collection of
code and data (sometimes referred to as functions and state) that is deployed using
cryptographically signed transactions on the blockchain network (e.g., Ethereum’s smart
contracts, Hyperledger Fabric’s chaincode). The smart contract is executed by nodes within the
blockchain network; all nodes that execute the smart contract must derive the same results from
the execution, and the results of execution are recorded on the blockchain.
