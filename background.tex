%% BACKGROUND SECTION 
\label{sec:overview}

In this Section, we present a brief overview of relevant background material that directly impacts our goal of utilizing blockchain
technology for COOP purposes.

Blockchains are tamper evident and tamper resistant digital ledgers implemented in a distributed fashion (i.e., without
a central repository) and thus do not have a central point of failure. At their basic level, they enable a community of
actors (users/machines/systems) to record transactions in a shared ledger within that community, such that under normal
operation of the blockchain network no transaction can be changed once published.

At a technical level, Blockchains are a distributed ledger comprised of blocks. Each block is comprised of a block
header containing metadata about the block, and block data containing a set of transactions and other related data.
Every block header (except the first) contains a cryptographic link to the previous
block’s header. Each transaction is digitally signed and added to the chain based on an {\it consensus} among the
participating entities (nodes/machines or users).

Trust within a blockchain network is enabled by four key characteristics of blockchain technology, which become
important in an adversarial scenario:
\begin{itemize}
\item {\bf Ledger} – the technology uses an append only ledger to provide full transactional history.
Unlike traditional databases, transactions and values in a blockchain are not overridden.
\item {\bf Secure} – blockchains are cryptographically secure, ensuring that the data contained
within the ledger has not been tampered with, and that the data within the ledger is
        attestable. Our design is {\it quantum-safe}, that is, it cannot be broken by a quantum computer in the future.
\item {\bf Shared} – the ledger is shared amongst multiple participants. This provides transparency, redundancy and
    resilience to failures/attacks across the node participants in the blockchain network.
\item {\bf Distributed} – the blockchain can be distributed. This allows for scaling the number of
nodes of a blockchain network to make it more resilient to attacks by bad actors. By
increasing the number of nodes, the ability for a bad actor to impact the consensus
protocol used by the blockchain is reduced.
\end{itemize}

There are two general high-level categories for blockchain approaches that have been identified: permissionless, and
permissioned. In a permissionless blockchain network anyone can read and write to the blockchain without authorization.
Permissioned blockchain networks limit participation to specific people or organizations and allow finer-grained
controls.  For our use case, we envision this to be a permissioned or a "consortium" design that explicitly allows
specific installations, such as government, defense or military entities to be part of the network.  The network
membership would be based off quantum-resistant asymmetric key based access control mechanisms that allow
delegation of roles and governance.

\subsection{Quantum-resistant distributed ledgers}

Most blockchain ledgers today rely on elliptic curve public-key cryptography (ECDSA) to generate digital signatures
which allow transactions to be verified securely. The most commonly used signature schemes today such as ECDSA, DSA and
RSA are theoretically vulnerable to quantum computing attack. RSA, DSA and ECDSA remain secure based upon the
computational difficulty of factorization of large integers, the discrete logarithm problem and the elliptic-curve
discrete logarithm problem. Shor’s quantum algorithm (see \cite{shor1997}) solves factorization of large integers and discrete
logarithms in polynomial time. Therefore, a quantum computer could theoretically reconstitute the private key given an
ECDSA public key. Smaller key sizes make ECDSA more vulnerable to quantum attack than RSA. It is expected that a 
1300 qubit quantum computer will be able to solve 228 bit ECDSA.

In this project, we will construct a functioning prototype of a quantum resistant blockchain ledger to counter the
potential advent of a non-linear quantum computing advance.  There are several important cryptographic systems which are
believed to be quantum-resistant: hash-based cryptography, code-based cryptography, lattice-based cryptography,
multivariate-quadratic-equations cryptography and secret-key cryptography. All these schemes are thought to resist both
classical and quantum computing attack given sufficiently long key sizes.  Forward secure hash-based digital signature
schemes exist with minimal security requirements that rely only upon the collision-resistance of a cryptographic hash
function. Changing the chosen hash function produces a new hash-based digital signature scheme. Hash-based digital
signatures are well studied and represent the primary candidate for post-quantum signatures in the future. As such they
are the chosen class of post-quantum signature for existing projects in this space such as \cite{qrl2016}.


\subsection{Consensus Models}

The consensus protocol is a key component of any blockchain. It determines how securely and
quickly blockchain validators reach consensus on the next block. Bitcoin uses Proof-of-Work which
has known in-efficiencies. For our permissioned blockchain network, we shall use Byzantine fault tolerant 
consensus mechanisms that can tolerant a configurable amount of failure. This allows the system
to reconfigure itself for a higher-trust and consensus agreement in the face of failure or attack, while
reverting back to normal operation under regular conditions.

An recent version of this is called the Practical Byzantine Fault Tolerance (PBFT), as given in \cite{castro1999}. In
PBFT, one node is elected as the “leader,” while the rest of the nodes are “validators.” Each round of PBFT consensus
involves two major phases: the prepare phase and the commit phase. In the prepare phase, the leader broadcasts its
proposal to all of the validators, who in turn broadcast their votes on the proposal to everyone else. The reason for
the rebroadcasting to all validators is that the votes of each validator need to be counted by all other validators. The
prepare phase finishes when more than $2f + 1$ consistent votes are seen, where $f$ is the number of malicious
validators, and the total number of validators plus the leader is $3f + 1$. The commit phase involves a similar vote
counting process, and consensus is reached when $2f + 1$ consistent votes are seen. Due to the rebroadcasting of votes
among validators, PBFT has $O(N)$ communication complexity. It is possible to configure this system for a higher
reliability in two ways -- (i) increasing the amount of consensus required, i.e., tuning the factor $f$ or (ii) biasing based
on a role -- that is certain actors in the network can have a higher "weight" factor.

Recent work by Team Rocket (see \cite{ava2018}) has made some breakthrough progress in consensus protocols which we aim
to incorporate into our design.

%
%\subsection{Distributed Ledgers}
%
%A ledger is a collection of transactions.  How would a ledger look ?
%Talk about Merkle trees and other data structures ?
%
%Centrally owned ledgers suffer from the problem of a single point of attack or failure.
%
%\subsection{Blocks}
%
%Blockchain network users submit candidate transactions to the blockchain network via software
%(desktop applications, smartphone applications, digital wallets, web services, etc.). The software
%sends these transactions to a node or nodes within the blockchain network. The chosen nodes
%may be non-publishing full nodes as well as publishing nodes. The submitted transactions are
%then propagated to the other nodes in the network, but this by itself does not place the transaction
%in the blockchain. For many blockchain implementations, once a pending transaction has been
%distributed to nodes, it must then wait in a queue until it is added to the blockchain by a
%publishing node.
%Transactions are added to the blockchain when a publishing node publishes a block. A block
%contains a block header and block data. The block header contains metadata for this block. The
%block data contains a list of validated and authentic transactions which have been submitted to
%the blockchain network. Validity and authenticity is ensured by checking that the transaction is
%correctly formatted and that the providers of digital assets in each transaction (listed in the
%transaction’s ‘input’ values) have each cryptographically signed the transaction. This verifies that
%the providers of digital assets for a transaction had access to the private key which could sign
%over the available digital assets. The other full nodes will check the validity and authenticity of
%all transactions in a published block and will not accept a block if it contains invalid transactions.
%It should be noted that every blockchain implementation can define its own data fields; however,
%many blockchain implementations utilize data fields like the following:
%• Block Header
%o The block number, also known as block height in some blockchain networks.
%o The previous block header’s hash value.
%o A hash representation of the block data (different methods can be used to
%accomplish this, such as a generating a Merkle tree (defined in Appendix B), and
%storing the root hash, or by utilizing a hash of all the combined block data).
%o A timestamp.
%
%The size of the block.
%o The nonce value. For blockchain networks which utilize mining, this is a number
%which is manipulated by the publishing node to solve the hash puzzle (see Section
%4.1 for details). Other blockchain networks may or may not include it or use it for
%another purpose other than solving a hash puzzle.
%• Block Data
%o A list of transactions and ledger events included within the block.
%o Other data may be present.
%
%\noindent {\bf Chaining of Blocks:} Blocks are chained together through each block containing the hash digest of the previous block’s header, thus forming the
%blockchain. If a previously published block were changed, it would have a different hash. This in turn would cause all subsequent
%blocks to also have different hashes since they include the hash of the previous block. This makes it possible to easily detect and reject
%altered blocks. Figure 3 shows a generic chain of blocks.
%
