%% BACKGROUND SECTION 
Blockchains are tamper evident and tamper resistant digital ledgers implemented
in a distributed fashion (i.e., without a central repository) and usually
without a central authority (i.e., a bank, company, or government). At their
basic level, they enable a community of users to record transactions in a
shared ledger within that community, such that under normal operation of the
blockchain network no transaction can be changed once published.

There are two general high-level categories for blockchain approaches that have
been identified: permissionless, and permissioned. In a permissionless
blockchain network anyone can read and write to the blockchain without
authorization. Permissioned blockchain networks limit participation to specific
people or organizations and allow finer-grained controls. Knowing the
differences between these two categories allows an organization to understand
which subset of blockchain technologies may be applicable to its needs.

Despite the many variations of blockchain networks and the rapid development of
new blockchain related technologies, most blockchain networks use common core
concepts.  Blockchains are a distributed ledger comprised of blocks. Each block
is comprised of a block header containing metadata about the block, and block
data containing a set of transactions and other related data. Every block
header (except for the very first block of the blockchain) contains a
cryptographic link to the previous block’s header. Each transaction involves
one or more blockchain network users and a recording of what happened, and it
is digitally signed by the user who submitted the transaction.


\begin{itemize}
\item Key properties of why blockchains. Ledger, repeatability, consensus.
\item Define how blockchains can help
\item Define the opportunity and problem statement.
\end{itemize}

\bf{KEY PROPERTIES OF BLOCKCHAIN THAT CAN HELP IN A DISASTER SCENARIO.}

 Prior to the use of blockchain
technology, this trust was typically delivered through intermediaries trusted by both parties.
Without trusted intermediaries, the needed trust within a blockchain network is enabled by four
key characteristics of blockchain technology, described below:
\begin{itemize}
\item \bf{Ledger} – the technology uses an append only ledger to provide full transactional history.
Unlike traditional databases, transactions and values in a blockchain are not overridden.
\item \bf{Secure} – blockchains are cryptographically secure, ensuring that the data contained
within the ledger has not been tampered with, and that the data within the ledger is
attestable.
\item \bf{Shared} – the ledger is shared amongst multiple participants. This provides transparency
across the node participants in the blockchain network.
\item \bf{Distributed} – the blockchain can be distributed. This allows for scaling the number of
nodes of a blockchain network to make it more resilient to attacks by bad actors. By
increasing the number of nodes, the ability for a bad actor to impact the consensus
protocol used by the blockchain is reduced.
\end{itemize}
