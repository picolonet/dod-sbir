%% BACKGROUND SECTION 
Blockchains are tamper evident and tamper resistant digital ledgers implemented
in a distributed fashion (i.e., without a central repository) and usually
without a central authority (i.e., a bank, company, or government). At their
basic level, they enable a community of users to record transactions in a
shared ledger within that community, such that under normal operation of the
blockchain network no transaction can be changed once published.

There are two general high-level categories for blockchain approaches that have
been identified: permissionless, and permissioned. In a permissionless
blockchain network anyone can read and write to the blockchain without
authorization. Permissioned blockchain networks limit participation to specific
people or organizations and allow finer-grained controls. Knowing the
differences between these two categories allows an organization to understand
which subset of blockchain technologies may be applicable to its needs.

Despite the many variations of blockchain networks and the rapid development of
new blockchain related technologies, most blockchain networks use common core
concepts.  Blockchains are a distributed ledger comprised of blocks. Each block
is comprised of a block header containing metadata about the block, and block
data containing a set of transactions and other related data. Every block
header (except for the very first block of the blockchain) contains a
cryptographic link to the previous block’s header. Each transaction involves
one or more blockchain network users and a recording of what happened, and it
is digitally signed by the user who submitted the transaction.


\begin{itemize}
\item Key properties of why blockchains. Ledger, repeatability, consensus.
\item Define how blockchains can help
\item Define the opportunity and problem statement.
\end{itemize}

\bf{KEY PROPERTIES OF BLOCKCHAIN THAT CAN HELP IN A DISASTER SCENARIO.}

 Prior to the use of blockchain
technology, this trust was typically delivered through intermediaries trusted by both parties.
Without trusted intermediaries, the needed trust within a blockchain network is enabled by four
key characteristics of blockchain technology, described below:
\begin{itemize}
\item \bf{Ledger} – the technology uses an append only ledger to provide full transactional history.
Unlike traditional databases, transactions and values in a blockchain are not overridden.
\item \bf{Secure} – blockchains are cryptographically secure, ensuring that the data contained
within the ledger has not been tampered with, and that the data within the ledger is
attestable.
\item \bf{Shared} – the ledger is shared amongst multiple participants. This provides transparency
across the node participants in the blockchain network.
\item \bf{Distributed} – the blockchain can be distributed. This allows for scaling the number of
nodes of a blockchain network to make it more resilient to attacks by bad actors. By
increasing the number of nodes, the ability for a bad actor to impact the consensus
protocol used by the blockchain is reduced.
\end{itemize}

While blockchain networks can be categorized aBlockchain networks can be categorized based on their permission model, which determines who
can maintain them (e.g., publish blocks). If anyone can publish a new block, it is permissionless.
If only particular users can publish blocks, it is permissioned. For our use case, we envision this to be a permissioned or a
"consortium" design that explicity allows specific installations, such as government, defense or military entities to be part of the network.
The network membership would be based off public key crytography based access control mechanisms that have been successfully used in the 
Internet today.

The focus of our design is to build a blockchain-based framework to store mission critical data.

\subsection{How Blockchains can utilize Quantum Computing:}


Refer section 3.1 on the strength of cryptographic hash functions.


\subsection{Distributed Ledgers}

A ledger is a collection of transactions.  How would a ledger look ?
Talk about Merkle trees and other data structures ?

Centrally owned ledgers suffer from the problem of a single point of attack or failure.

\subsection{Blocks}

Blockchain network users submit candidate transactions to the blockchain network via software
(desktop applications, smartphone applications, digital wallets, web services, etc.). The software
sends these transactions to a node or nodes within the blockchain network. The chosen nodes
may be non-publishing full nodes as well as publishing nodes. The submitted transactions are
then propagated to the other nodes in the network, but this by itself does not place the transaction
in the blockchain. For many blockchain implementations, once a pending transaction has been
distributed to nodes, it must then wait in a queue until it is added to the blockchain by a
publishing node.
Transactions are added to the blockchain when a publishing node publishes a block. A block
contains a block header and block data. The block header contains metadata for this block. The
block data contains a list of validated and authentic transactions which have been submitted to
the blockchain network. Validity and authenticity is ensured by checking that the transaction is
correctly formatted and that the providers of digital assets in each transaction (listed in the
transaction’s ‘input’ values) have each cryptographically signed the transaction. This verifies that
the providers of digital assets for a transaction had access to the private key which could sign
over the available digital assets. The other full nodes will check the validity and authenticity of
all transactions in a published block and will not accept a block if it contains invalid transactions.
It should be noted that every blockchain implementation can define its own data fields; however,
many blockchain implementations utilize data fields like the following:
• Block Header
o The block number, also known as block height in some blockchain networks.
o The previous block header’s hash value.
o A hash representation of the block data (different methods can be used to
accomplish this, such as a generating a Merkle tree (defined in Appendix B), and
storing the root hash, or by utilizing a hash of all the combined block data).
o A timestamp.

The size of the block.
o The nonce value. For blockchain networks which utilize mining, this is a number
which is manipulated by the publishing node to solve the hash puzzle (see Section
4.1 for details). Other blockchain networks may or may not include it or use it for
another purpose other than solving a hash puzzle.
• Block Data
o A list of transactions and ledger events included within the block.
o Other data may be present.

\noindent \bf{Chaining of Blocks:} Blocks are chained together through each block containing the hash digest of the previous block’s header, thus forming the
blockchain. If a previously published block were changed, it would have a different hash. This in turn would cause all subsequent
blocks to also have different hashes since they include the hash of the previous block. This makes it possible to easily detect and reject
altered blocks. Figure 3 shows a generic chain of blocks.

\subsection{Consensus Models}
