\label{sec:design}
This section describes the technical design for achieving our Phase I/II objectives. :vs

The principal components of a secure and performant Blockchain system include a decentralized and distributed legder, a
network stack, a consensus mechanism and possibly a smart contract system in a turing complete language. The reader is
referred to some prior research that capture contemporary design concepts in this space, namely \cite{micali16,
garay2015, ava2018}. Also we refer the reader to Picolo Labs whitepaper for a planet-level robut decentralized database
design \cite{picolo}.

ORIG:
PHASE I: An engineering prototype will be constructed using a quantum-resistant blockchain framework designed to support
COOP. The engineering prototype will be comprised from a set of twenty independent systems connected logically through
the blockchain framework. The performer will conduct a series of experiments on the engineering prototype while
simulating the two different network environments outlined above. In the experiments, simulated users will
stochastically generate a large number of documents that are passed to one of the blockchain systems for replication to
the other blockchain systems. For the simulated global WAN, these documents should vary in size from 20 kB to 10 MB. For
the WAN that includes a simulated satellite link, these documents should vary in size from 20 kB to 500 kB. In addition
to the network factor, the performer’s experimental design should include factors for: • degree of sudden physical
compromise of the system (i.e., a percentage of systems whose total sudden physical loss is simulated), • degree of
cyber-compromise (i.e., a percentage of systems whose compromise is simulated), and • degree of denial-of-service (i.e.,
levels of a cyber adversary’s denial-of-service efforts). The performer will observe the accuracy of the preserved
documents as well as the overall recall of the generated documents at the end of each experiment. The performer will
also collect the total quantity and overall characteristics of the network traffic generated by the system. The
performer will verify and validate that data block generation is secure and that the generated data blocks are immutable
and auditable. The phase I deliverable is a report, delivery of collected data, and a demonstration of the engineering
prototype system.

\subsubsection{Network stack}

We shall use a p2p network stack designed for redudancy and resiliency to failures. There is wide range of research
literature to draw a design from.  The Freenet \cite{freenet_thesis, Clarke_2001} and the Gnutella \cite{Gnutella} p2p
systems were popular in the previous decade for file sharing. Both systems were designed for sharing of large files over
a longer duration of time. Content reliabilty including lookup reliability and network latency goals were necessary in
this enviroment. The second generation of p2p systems include research driven projects such as Chord \cite{Stoica_2001},
Content Addressable Network (CAN) \cite{Ratnasamy_2001}, Pastry \cite{Rowstron_2001}, Tapestry \cite{tapestry2004} and
Kademlia. Chord along with CAN, Tapestry and Pastry developed the concept of distributed hash tables (DHTs) as a
fundamental mechanism for content-based addressing.

Our design builds upon this prior work by using a quantum-resistant distributed hash table for
publish/subscribe of services will allow for database services to move from one compute node to another in the face of
failures. A heatbeat based model for node aliveness and discovery could accomplish the goals of resilience to diverse
network conditions and coordinated failures. Our design space includes the possibility of utilizing error-correcting or
detecting codes such as erasure codes / fountain codes (\cite{byers1998}, \cite{hu2013}) or other ways of adding
redudancy into the application layer protocols \cite{bloxroute}.

\subsubsection{Authenticaiton and access control}

Since this is a permissioned blockchain network, each node shall be instantiated with a agency certificate and a node
certificate. The agency or office certificate gives the local administrator certain privileges in terms of deactiviating
potentially malicious nodes or provisionin additional capacity into the system. Each transaction into the blockchain
ledger shall be authenticated by the proposer and validated using a consensus mechanism. The blockchain acts as a
distriubted computer with its own consensus mechanism that is able to function in various modes: a normal mode where
operations are allowed as per regular consensus mechanisms and a safe mode when the system is under attack. Using a
system of smart contracts (Phase II work) the system is collectively able to transition between such modes intelligently.

\subsubsection{Distributed Ledger}

Our design of the distributed ledger will make use of concepts from the Trillian project which builds upon Merkle trees
and Verifiable Data structures \cite{verifiable2015}. This allows the system to scale to potentially billions of
indexable records should the need arise.

The blockchain data store is built upon the Verifiable Data structures concept \cite{verfiable2015}. The idea is that each
system in the chain of mission criticial IT systems, would be able to submit transactions which get recorded on the
blockchain as append-only logs. Since any of these systems could be compromised or attacked, we assume a "trust but
verify" philosophy that also underlies such data structures. The full design of how the blockchain uses this data
structures shall be completed during Phase II of the project.

\subsubsection{Proof of storage}

While not explicitly a part of Phase I, the system incorporates succint and zero knowledge proofs of storage. Proofs
of data and storage are cryptographic proofs constructed by participant nodes. These are quantum resistant and can be
used to prove the existing (or not) of data and node availability. While not an explicit goal in Phase I or II, our
design allows us to use such strong cryptographic constructs to make the system highly resilient to any coordinated
attacker \cite{ben2019}.

\subsubsection{Design of a Recovery mode}

Our blockchain framework allows us to design a recovery mode which is a protocol that will get activiated based on a 
smart contract and consensus system which puts the blockchain into a disaster recovery mode. A robust, fast and turing
complete smart contract system can be used for this design \cite{kalodner2018}. This mode can put in 
stricter controls and automatic locking of certain types of data silos, for example. Over time, this could be programmed
for other mission specific recovery mode behaviors.
