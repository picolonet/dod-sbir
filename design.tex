This section describes the technical objectives and a design for achieving them in Phase I of the SBIR project. The
technical goals for Phase I are the following:

\begin{enumerate}

\item {\bf Design:} Complete a design of a blockchain-based quantum resistant system for mission critical operation and
COOP goals. The milestone consists of a design document with high-level specification of each component of the
blockchain system, including design of the ledger, the consensus mechanism, database replication, data query and other
supported operations.

\item {\bf Internal Testbed :} Build a testbed using hosted containers and simulated networks for iterated model of code development and testing
before moving to the real test. This capability will be important for rapid prototyping, testing and development in-house.

\item {\bf Simulation readiness :} The prototype blockchain would have undergone enough internal testing to be ready for
the simulation run. This milestone also includes implemenation of the simulation environment, including creation of
synthetic users, test cases, instrumentation logic measuring real-time and statistical performance properties and
related software components. As given in the topic announcement, the simulation consists of conducting a set of
experiments to measure variuos aspects of the blockchain system. 
 
\item {\bf Simulation execution :} The simulation includes experiments that mimic different kinds of disruptions, such as
testing for network partitions, coordinated failures of network and nodes (including denial of service attack patterns)
and measurement of performance under high latency and packet losses. Some of the characteristics that we shall measure
include but not limited to:

\begin{enumerate}

put some metrics here.

\end{enumerate}

\item {Readiness for Phase II:} The system should be implemented and designed at a level that makes it ready for Phase
II development and progress.
\end{enumerate}


ORIG:
PHASE I: An engineering prototype will be constructed using a quantum-resistant blockchain framework designed to support
COOP. The engineering prototype will be comprised from a set of twenty independent systems connected logically through
the blockchain framework. The performer will conduct a series of experiments on the engineering prototype while
simulating the two different network environments outlined above. In the experiments, simulated users will
stochastically generate a large number of documents that are passed to one of the blockchain systems for replication to
the other blockchain systems. For the simulated global WAN, these documents should vary in size from 20 kB to 10 MB. For
the WAN that includes a simulated satellite link, these documents should vary in size from 20 kB to 500 kB. In addition
to the network factor, the performer’s experimental design should include factors for: • degree of sudden physical
compromise of the system (i.e., a percentage of systems whose total sudden physical loss is simulated), • degree of
cyber-compromise (i.e., a percentage of systems whose compromise is simulated), and • degree of denial-of-service (i.e.,
levels of a cyber adversary’s denial-of-service efforts). The performer will observe the accuracy of the preserved
documents as well as the overall recall of the generated documents at the end of each experiment. The performer will
also collect the total quantity and overall characteristics of the network traffic generated by the system. The
performer will verify and validate that data block generation is secure and that the generated data blocks are immutable
and auditable. The phase I deliverable is a report, delivery of collected data, and a demonstration of the engineering
prototype system.

\subsection{System Design}

\subsubsection{Network stack}

erasure codes, fountain codes (ref Harmony).

\subsubsection{Node discovery and aliveness}

\subsubsection{Authenticaiton and access control}

Since this is a permissioned blockchain network, each node shall be instantiated with a agency certificate and a node
certificate. The agency or office certificate gives the local administrator certain privileges in terms of deactiviating
potentially malicious nodes or provisionin additinal capacity into the system.

\subsubsection{Distributed Ledger}

How will we use Trillian or Merkle trees.

The blockchain data store is built upon the Verifiable Data structures concept \cite{verfiable2015}. The idea is that each
system in the chain of mission criticial IT systems, would be able to submit transactions which get recorded on the
blockchain as append-only logs. Since any of these systems could be compromised or attacked, we assume a "trust but
verify" philosophy that also underlies such data structures. The full design of how the blockchain uses this data
structures shall be completed during Phase II of the project.

\subsubsection{Proof of storage}

While not explicitly a part of Phase I, the system incorporates succint and zero knowledge proofs of storage. Add
references.


\subsubsection{Design of a Recovery mode:}

A protocol that will get activiated based on a smart contract and consensus system which puts the blockchain into a
disaster recovery mode.
What happens in the recovery mode of operation vs normal mode ?
