\label{sec:design} This section describes the technical design for achieving our Phase I/II objectives. The principal
components of a secure and performant Blockchain system include a decentralized and distributed quantum-resistant ledger, a network stack,
a consensus mechanism and possibly a smart contract system in a Turing complete language. The reader is referred to some
prior research that capture contemporary design concepts in this space, namely \cite{micali16, garay2015, ava2018}. Also
we refer the reader to Picolo Labs white-paper for a planet-level robust decentralized database design (see
\cite{picolo2018}).

\subsection{Network stack}

We shall use a peer-to-peer (p2p) network stack designed for redundancy and resiliency to failures. There is wide range of research
literature to draw a design from.  The Freenet \cite{freenet_thesis, Clarke_2001} and the Gnutella \cite{Gnutella} p2p
systems were popular in the previous decade for file sharing under highly unreliable conditions.
Content reliability including lookup reliability and network latency goals were necessary in
this environment. The second generation of p2p systems include research projects such as Chord \cite{Stoica_2001},
Content Addressable Network (CAN) \cite{Ratnasamy_2001}, Pastry \cite{Rowstron_2001}, Tapestry \cite{tapestry2004} and
Kademlia. Chord along with CAN, Tapestry and Pastry developed the concept of distributed hash tables (DHTs) as a
fundamental mechanism for content-based addressing.

Our design builds upon this prior work by using a quantum-resistant distributed hash table for
publish/subscribe of services will allow for database services to move from one compute node to another in the face of
failures. A heartbeat protocol for node aliveness and discovery could accomplish the goals of resilience to diverse
network conditions and coordinated failures. Our design space includes the possibility of utilizing error-correcting or
detecting codes such as erasure codes / fountain codes (\cite{byers1998}, \cite{hu2013}) or other ways of adding
redundancy into the application layer protocol (see \cite{bloxroute}).

\subsection{Authentication and access control}

Since this is a permissioned blockchain network, each node shall be instantiated with a agency certificate and a node
certificate. The agency or office certificate gives the local administrator certain privileges in terms of deactivating
potentially malicious nodes or provisioning additional capacity into the system. Each transaction into the blockchain
ledger shall be authenticated by the proposer and validated using a consensus mechanism. The blockchain acts as a
distributed computer with its own consensus mechanism that is able to function in various modes: a normal mode where
operations are allowed as per regular consensus mechanisms and a safe mode when the system is under attack. Using a
system of smart contracts (Phase II work) the system is collectively able to transition between such modes intelligently.

\subsection{Distributed Ledger}

Our design of the distributed ledger will make use of concepts from the Trillian project which builds upon Merkle trees
and Verifiable Data structures (see \cite{verifiable2015}). This allows the system to scale to potentially billions of
indexable records should the need arise.

The blockchain data store is built upon the Verifiable Data structures concept (\cite{verifiable2015}). The idea is that each
system in the chain of mission critical IT systems, would be able to submit transactions which get recorded on the
blockchain as append-only logs. Since any of these systems could be compromised or attacked, we assume a "trust but
verify" philosophy that also underlies such data structures. The full design of how the blockchain uses this data
structures shall be completed during Phase II of the project.

\subsection{Proof of storage}

While not explicitly a part of Phase I, the system incorporates succinct and zero knowledge proofs of storage. Proofs
of data and storage are cryptographic proofs constructed by participant nodes. These are quantum resistant and can be
used to prove the existing (or not) of data and node availability. While not an explicit goal in Phase I or II, our
design allows us to use such strong cryptographic constructs to make the system highly resilient to any coordinated
attacker (see \cite{ben2019}).

\subsection{Design of a Recovery mode}

Our blockchain framework allows us to design a recovery mode which is a protocol that will get activated based on a 
smart contract and consensus system which puts the blockchain into a disaster recovery mode. A robust, fast and Turin
complete smart contract system can be used for this design (see \cite{kalodner2018}). This mode can put in 
stricter controls and automatic locking of certain types of data silos, for example. Over time, this could be programmed
for other mission specific recovery mode behaviors.
